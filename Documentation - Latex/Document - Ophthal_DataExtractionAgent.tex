\documentclass[a4paper,12pt]{article}
\usepackage[utf8]{inputenc}
\usepackage{amsmath}
\usepackage{amssymb}
\usepackage{hyperref}
\usepackage{listings}
\usepackage{xcolor}

\hypersetup{
    colorlinks=true,
    linkcolor=blue,
    filecolor=magenta,      
    urlcolor=cyan,
}

\lstset{
    basicstyle=\ttfamily\small,
    breaklines=true,
    postbreak=\mbox{\textcolor{red}{$\hookrightarrow$}\space},
    breakatwhitespace=true,
    frame=single,
    language=Python
}

\title{Project Documentation: Ophthalmology Data Extraction}
\author{}
\date{}

\begin{document}

\maketitle

\tableofcontents
\newpage

\section{Overview}
This project involves extracting question-answer pairs related to ophthalmology from multiple web pages and saving them in a CSV file. The script uses Python's \texttt{requests} library for web scraping and \texttt{BeautifulSoup} for parsing HTML content. It includes error handling, data validation, and ensures the extracted data is relevant to ophthalmology.

\section{Requirements}
\begin{itemize}
    \item Python 3.x
    \item aiohttp library
    \item \texttt{requests} library
    \item \texttt{beautifulsoup4} library
    \item \texttt{csv} module (part of Python's standard library)
    
\end{itemize}

\section{Setup Instructions}
\begin{enumerate}
    \item \textbf{Install Python}: Ensure Python 3.x is installed on your system. You can download it from \href{https://www.python.org/}{python.org}.
    \item \textbf{Install Required Libraries}: Install the necessary Python libraries using pip:
    \begin{lstlisting}[language=bash]
pip install aiohttp beautifulsoup4 requests
    \end{lstlisting}
\end{enumerate}

\section{Script Explanation}

\subsection{Fetching Web Pages}
The \texttt{fetch\_page} function retrieves the content of a given URL and handles potential request errors.

\begin{lstlisting}
import requests
from requests.exceptions import RequestException

def fetch_page(url):
    headers = {
        "User-Agent": "Mozilla/5.0 (Windows NT 10.0; Win64; x64) AppleWebKit/537.36 (KHTML, like Gecko) Chrome/58.0.3029.110 Safari/537.3"
    }
    try:
        response = requests.get(url, headers=headers)
        response.raise_for_status()
        return response.content
    except RequestException as e:
        print(f"Request failed for {url}: {e}")
        return None
\end{lstlisting}

\subsection{Parsing Question-Answer Pairs}
The \texttt{parse\_questions\_answers} function parses the HTML content to extract question-answer pairs. It ensures that only relevant questions (related to ophthalmology) are included.

\begin{lstlisting}
from bs4 import BeautifulSoup

def parse_questions_answers(soup):
    qa_pairs = []
    questions = soup.find_all('h1')
    for question in questions:
        answer = question.find_next('p')
        question_text = question.get_text(strip=True)
        answer_text = answer.get_text(strip=True) if answer else 'No answer found'
        if is_valid_ophthalmology_question(question_text):
            qa_pairs.append((question_text, answer_text))
    return qa_pairs

def is_valid_ophthalmology_question(question):
    ophthalmology_keywords = ['eye', 'vision', 'optometrist', 'ophthalmologist', 'eye health']
    return any(keyword.lower() in question.lower() for keyword in ophthalmology_keywords)
\end{lstlisting}

\subsection{Saving Data to CSV}
The \texttt{save\_to\_csv} function saves the extracted question-answer pairs to a CSV file.

\begin{lstlisting}
import csv

def save_to_csv(data, filename):
    with open(filename, mode='w', newline='', encoding='utf-8') as file:
        writer = csv.writer(file)
        writer.writerow(["Question", "Answer"])
        writer.writerows(data)
\end{lstlisting}

\subsection{Main Function}
The \texttt{main} function coordinates the scraping process, iterating over multiple URLs and aggregating the results.

\begin{lstlisting}
def main():
    urls = [
        "https://www.healthline.com/health/eye-health/optometrist-vs-ophthalmologist#ophthalmologist",
        # Add more URLs here
    ]
    
    all_qa_pairs = []

    for url in urls:
        page_content = fetch_page(url)
        if page_content:
            soup = BeautifulSoup(page_content, 'html.parser')
            qa_pairs = parse_questions_answers(soup)
            all_qa_pairs.extend(qa_pairs)
    
    save_to_csv(all_qa_pairs, "vision_health_qa.csv")
    print("Data extraction and saving completed.")

if __name__ == "__main__":
    main()
\end{lstlisting}

\subsection{Main Function}
\begin{lstlisting}[language=Python, caption={Main Function}]
async def main(api_key=None):
    '''Main function to coordinate the scraping process and save the results.'''
    urls = [
        "https://www.healthline.com/health/eye-health/optometrist-vs-ophthalmologist",
        "https://www.healthdirect.gov.au/ophthalmologist",
        "https://www.webmd.com/eye-health/default.htm",
        "https://www.aao.org/eye-health",
        "https://www.med.unc.edu/ophth/for-patients/eye-diseases-and-disorders/",
        "https://healthcare.utah.edu/moran/services",
        "https://www.cdc.gov/vision-health/about-eye-disorders/index.html",
        "https://medlineplus.gov/eyesandvision.html",
        "https://www.news-medical.net/condition/Eye-Health"
    ]
    
    async with aiohttp.ClientSession() as session:
        tasks = [fetch_page(session, url) for url in urls]
        pages_content = await asyncio.gather(*tasks)
        
        all_qa_pairs = []
        for content in pages_content:
            if content:
                soup = BeautifulSoup(content, 'html.parser')
                qa_pairs = parse_questions_answers(soup, api_key=api_key)
                all_qa_pairs.extend(qa_pairs)
        
        save_to_csv(all_qa_pairs, "ophthalmology_qa.csv")
        print("Data extraction and saving completed.")

# Run the main function
api_key = 'your_openai_api_key_here'  # Replace with your actual OpenAI API key if available
asyncio.run(main(api_key=api_key))
\end{lstlisting}

\section{Instructions for the Intern}
\begin{enumerate}
    \item \textbf{Understand the Current Code}: Thoroughly read and understand the existing code. Pay special attention to the functions and how they interact.
    \item \textbf{Add More URLs}: Expand the \texttt{urls} list with additional URLs that contain relevant ophthalmology information.
    \item \textbf{Improve Validation}:\\
     Enhance the \texttt{is\_valid\_ophthalmology\_question} function to better filter relevant questions. Consider using more advanced NLP techniques if necessary.
    \item \textbf{Error Handling}: Improve error handling to make the script more robust. Consider logging errors to a file for better traceability.
    \item \textbf{Documentation}: Maintain detailed documentation of any changes made. Ensure comments in the code are clear and descriptive.
    \item \textbf{Code Testing}: Test the code with various URLs to ensure it works correctly and handles edge cases.
    \item \textbf{Version Control}: Use a version control system (e.g., Git) to track changes and collaborate effectively.
    \item \textbf{Optimization}: Optimize the code for performance, especially if dealing with a large number of URLs or large HTML content.
    \item \textbf{User Interface}: If time permits, consider creating a simple user interface (CLI or GUI) to allow non-technical users to add URLs and run the script.
    \item \textbf{Regular Updates}: Regularly update the script to adapt to changes in the HTML structure of target websites.
\end{enumerate}

\section{Future Enhancements}
\begin{itemize}
    \item \textbf{Advanced Data Processing}: Implement advanced data processing techniques to refine the extracted data.
    \item \textbf{Machine Learning Integration}: Use machine learning models to automatically classify and validate question-answer pairs.
    \item \textbf{Scalability}: Make the script scalable to handle a larger volume of data and concurrent requests.
\end{itemize}

\end{document}
